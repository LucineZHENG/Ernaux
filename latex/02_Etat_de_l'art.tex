L’analyse du style littéraire à l’aide d’outils numériques constitue aujourd’hui un champ de recherche dynamique au croisement de la linguistique computationnelle et des études littéraires. Les approches combinant méthodes quantitatives et interprétation stylistique ont permis d’éclairer de nouvelles dimensions formelles dans des œuvres appartenant à diverses époques et traditions. Dans cette section, nous présentons plusieurs travaux qui ont marqué ce domaine et dont les apports méthodologiques nourrissent notre propre démarche.\\

\par Le cadre théorique d’analyse du style littéraire dans les romans français du XIX\textsuperscript{e} siècle, élaboré par Lu Yuanfeng dans sa thèse de doctorat, démontre de manière convaincante la capacité des schémas syntaxiques à distinguer les styles littéraires. En combinant l’analyse syntaxique en dépendances avec la modélisation statistique via le système UDPipe, il construit un modèle quantitatif innovant permettant non seulement de valider mais aussi de remettre en question certaines conceptions traditionnelles sur le style des écrivains. Parallèlement, Garric et Maurel-Indart, dans leur ouvrage \textit{Vers une automatisation de l’analyse textuelle}, ancrent leur réflexion autour de la problématique interdisciplinaire suivante : dans quelle mesure un style peut-il être modélisé et reconnu automatiquement ? À travers une étude de cas portant sur \textit{La Princesse de Clèves}, ils explorent systématiquement les possibilités théoriques d’une telle analyse automatisée du style.\\

\par Ces recherches démontrent non seulement l’efficacité des approches syntaxiques et statistiques dans l’identification des styles, mais elles constituent également une base méthodologique solide pour la caractérisation de l’\textit{écriture plate} dans un cadre de traitement automatique des langues (TAL). En particulier, le travail de modélisation de Lu sur les styles narratifs du XIX\textsuperscript{e} siècle montre comment les structures syntaxiques peuvent révéler des différences systématiques entre les auteurs — un apport précieux pour la compréhension des traits de simplicité et de dépouillement caractéristiques de l’écriture d’Annie Ernaux. De plus, la question de la « modélisabilité » du style posée par Garric et Maurel-Indart résonne avec notre problématique centrale : dans quelle mesure un style d’écriture peut-il être formalisé, identifié par des moyens computationnels, et ainsi se voir attribuer une valeur esthétique distincte ?\\

\par Or, bien que de nombreuses études aient exploré les liens entre style, syntaxe, lexique ou figures rhétoriques, la modélisation systématique de l’\textit{écriture plate} en tant que genre stylistique spécifique demeure encore largement absente. Dans le contexte de la littérature française contemporaine, les recherches sur l’\textit{écriture plate} se concentrent principalement sur des analyses philosophiques ou narratives, sans proposer de description opérationnelle de ses caractéristiques linguistiques.\\

\par Dans ce contexte, notre étude postule qu’une analyse stylistique efficace doit articuler plusieurs dimensions : une mesure précise des indicateurs quantitatifs (TTR, longueur moyenne des phrases, diversité syntaxique, etc.), une validation croisée par des analyses qualitatives, ainsi qu’une mise en perspective comparative avec d’autres auteurs du même contexte littéraire. L’objectif est de déterminer si l’\textit{écriture plate} constitue un registre stylistique identifiable, susceptible d’être situé dans une généalogie esthétique cohérente.

