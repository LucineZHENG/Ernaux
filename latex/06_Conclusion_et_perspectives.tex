Après cette étude, nous pouvons conclure que les œuvres d'Annie Ernaux ne sont pas absolument "binarisées". Bien que nous puissions observer des différences entre ses œuvres précoces et tardives dans certains aspects, tels que la longueur des phrases et la richesse lexicale, ces différences ne sont pas absolues. En effet, certaines œuvres tardives montrent une tendance vers un style plus simple et minimaliste (comme \textit{Passion simple} et \textit{Je ne suis pas sortie de ma nuit}), tandis que d'autres adoptent des phrases plus longues (comme \textit{Les Années} et \textit{Retour à Yvetot}). De plus, bien que les œuvres d'Ernaux présentent des évolutions en termes de syntaxe et de rhétorique, certains éléments restent présents tout au long de ses œuvres. Par exemple, l'utilisation fréquente de la répétition semble être un moyen pour elle de s'explorer elle-même et de revenir sur son passé. Ces évolutions suggèrent que le travail d'Ernaux présente une tendance à la "diversification" plutôt qu'une séparation stricte en deux catégories. Par conséquent, nous pouvons considérer que son changement de style est un processus graduel plutôt qu'une rupture totale.\\

Concernant l’analyse du corpus externe, les œuvres d’Annie Ernaux manifestent généralement une caractéristique de "style réaliste", en particulier dans la description de la vie quotidienne, des émotions personnelles et des contextes sociaux. Elle utilise un langage simple et direct. Cependant, ce "style réaliste" n’est pas propre à Ernaux ; il s'agit d'une caractéristique courante parmi ses contemporains. De nombreux écrivains, en particulier ceux représentant la littérature autobiographique et réaliste, ont tendance à utiliser un langage simple et direct pour décrire des expériences personnelles et des réalités sociales. Par exemple, des auteurs tels qu'Albert Camus ou Michel Houellebecq utilisent également un style d'écriture épuré. Toutefois, le "style réaliste" d'Ernaux se distingue par sa finesse émotionnelle et son observation unique des changements sociaux, ce qui lui confère une dimension personnelle et distinctive.\\

Quant aux limites de cette recherche, il convient de noter que l’analyse du vocabulaire et de la structure syntaxique n’a pas été assez approfondie. En effet, la structure des phrases est extrêmement complexe, car elle dépend non seulement de la combinaison des mots, mais aussi des relations et interactions entre les différents éléments de la phrase. Dans cette étude, nous nous sommes principalement concentrés sur la longueur moyenne des phrases et certains indicateurs de base concernant les classes grammaticales des mots. Cependant, ces méthodes sont trop simplistes pour rendre compte de la multidimensionnalité de la structure des phrases. Pour mieux comprendre les couches structurelles des phrases et les relations internes entre leurs composants, nous avons besoin d'outils d'analyse syntaxique plus précis. Comme l'exemple donné par Yuanfeng Lu dans \cite{lu2021caracterisation}\textit{Caractérisation des styles littéraires par l'extraction automatique des patrons syntaxiques dans des romans français du 19ème siècle}, l'utilisation d'un arbre syntaxique permet de rendre plus claires les structures et les relations hiérarchiques entre les différents composants de la phrase. Par exemple, dans les phrases complexes, la combinaison des verbes, noms et adjectifs ne suit pas une simple superposition linéaire, mais crée une structure hiérarchique par des règles grammaticales spécifiques, une dimension que nous n'avons pas suffisamment explorée.\\

Enfin, comme l’indique Garric, N. dans \cite{garric2011vers} "Comme il a été souligné, un référentiel conçu sur la base des seuls marqueurs linguistiques serait insuffisant pour l’analyse stylistique notamment déterminée par des interprétants génériques". Dans notre étude, le corpus externe utilisé comprend de nombreux romans, certains avec des caractéristiques autobiographiques, tandis que le corpus interne d'Ernaux est principalement composé de romans autobiographiques, mais aussi de journaux (comme \textit{Journal du dehors} et \textit{Regarde les lumières mon amour}) et de mémoires (comme \textit{Les Années}). Les différences de genre entre ces deux corpus peuvent avoir un impact potentiel sur les résultats de l’analyse.\\


Pour les recherches futures, je souhaiterais explorer davantage les caractéristiques linguistiques du corpus externe, tout en élargissant ce dernier pour étudier en profondeur le style d'écriture des autres auteurs contemporains. Parallèlement, j’espère pouvoir approfondir l’étude des styles d’écriture des écrivaines contemporaines en me basant sur les recherches liées à la longueur des phrases dans leurs œuvres.
