\section{Le corpus d’étude}

Le corpus utilisé dans cette étude comprend deux ensembles. Le premier est un corpus interne constitué de l’ensemble des œuvres publiées par Annie Ernaux, couvrant une période allant de \textit{Les Armoires vides} (1974) à \textit{Le Jeune Homme} (2022). Afin d’observer l’évolution stylistique de son écriture, nous avons classé ces textes en différentes catégories génériques, notamment les romans de jeunesse, les journaux intimes ou extimes, et les récits que nous supposons appartenir à l’esthétique de l’\textit{écriture plate}. Cette catégorisation est fondée à la fois sur des critères éditoriaux, critiques et stylistiques. Le tableau \ref{table:corpus_interne} présente l’ensemble des œuvres d’Annie Ernaux et leur classification dans le corpus interne.|\\


Le second corpus est un corpus externe, composé de 15 œuvres de 9 auteurs qui représentent des courants littéraires importants de la littérature francophone de l'après-guerre jusqu'aux années 1970 : l'existentialisme (Camus, Sartre), le Nouveau Roman (Robbe-Grillet, Sarraute, Simon, etc.), ainsi que des auteurs à la profondeur historique et culturelle telle que Yourcenar. Parmi les auteurs masculins figurent Albert Camus, Jean-Paul Sartre, Michel Tournier, Alain Robbe-Grillet, Robert Pinget, et Claude Simon. Les auteures féminines incluent Nathalie Sarraute, Marguerite Yourcenar, et Moira. Leurs œuvres serviront de points de comparaison stylistique pour situer Ernaux dans le paysage littéraire de son époque.Le tableau \ref{table:corpus_externes} présente les titres des œuvres choisies pour le corpus externe, ainsi que les auteurs associés.




\begin{table}[h!]
\centering
\begin{tabular}{|l|l|l|}
\hline
\textbf{Titre}                       & \textbf{Année} & \textbf{Catégorie}     \\ \hline
Les Armoires vides                   & 1974           & De jeunesse            \\ \hline
Ce qu'ils disent ou rien             & 1977           & De jeunesse            \\ \hline
La Femme gelée                       & 1981           & De jeunesse            \\ \hline
La Place                             & 1983           & Plate ?                \\ \hline
Une Femme                            & 1987           & Plate ?                \\ \hline
Passion simple                       & 1991           & Plate ?                \\ \hline
Journal du dehors                    & 1993           & Journal extime         \\ \hline
Je ne suis pas sortie de ma nuit     & 1997           & Journal                \\ \hline
La Honte                             & 1997           & Plate ?                \\ \hline
L'événement                          & 2000           & Plate ?                \\ \hline
Se Perdre                            & 2001           & Plate ?                \\ \hline
L'Autre Fille                        & 2001           & Plate ?                \\ \hline
L'Occupation                         & 2002           & Plate ?                \\ \hline
Les Années                           & 2008           & Plate ?                \\ \hline
L'Atelier noir                       & 2011           & Journal                \\ \hline
Retour à Yvetot                      & 2013           & Plate ?                \\ \hline
Regarde les lumières mon amour       & 2014           & Journal extime         \\ \hline
Mémoire de fille                     & 2016           & Plate ?                \\ \hline
Le jeune homme                       & 2022           & Plate ?                \\ \hline
\end{tabular}
\caption{Corpus interne (œuvres d'Annie Ernaux)}
\label{table:corpus_interne}
\end{table}

\begin{table}[h!]
\centering
\begin{tabular}{|l|l|}
\hline
\textbf{Titre}                                        & \textbf{Auteur}     \\ \hline
C\_1951\_Camus\_Chute.txt                              & Camus               \\ \hline
C\_1964\_Sartre\_Mots.txt                              & Sartre              \\ \hline
C\_1967\_Tournier\_Vendredi.txt                        & Tournier            \\ \hline
C\_1975\_Tournier\_Meteores.txt                        & Tournier            \\ \hline
C\_1950-1972\_Green\_Moira.txt                         & Green               \\ \hline
L\_1959\_Sarraute\_Planetarium.txt                     & Sarraute            \\ \hline
L\_1967\_Simon\_Histoire.txt                           & Simon               \\ \hline
L\_1955\_RobbeGrillet\_Jalousie.txt                    & Robbe-Grillet       \\ \hline
L\_1962\_Pinget\_Inquisitoire.txt                      & Pinget              \\ \hline
L\_1976\_Sarraute\_Disent.txt                          & Sarraute            \\ \hline
yourcenar\_homme-obscur\_une-belle-matinee.txt         & Yourcenar           \\ \hline
yourcenar\_souvenirs-pieux.txt                         & Yourcenar           \\ \hline
yourcenar\_archives-du-nord.txt                        & Yourcenar           \\ \hline
yourcenar\_oeuvre-au-noir.txt                          & Yourcenar           \\ \hline
yourcenar\_memoires-hadrien.txt                        & Yourcenar           \\ \hline
\end{tabular}
\caption{Corpus externe (auteurs contemporains)}
\label{table:corpus_externes}
\end{table}

\clearpage

\section{Approche méthodologique}
Tout d'abord, nous procédons au traitement du corpus en affichant les données existantes et en les nettoyant afin d'assurer l'exactitude et la cohérence des données. Ensuite, nous entamons la phase de recherche proprement dite.\\

Premièrement, au niveau lexical, nous identifierons le thème de chaque œuvre afin de comprendre le contenu principal de chaque texte. Nous analyserons également les verbes et les auxiliaires utilisés dans chaque œuvre, en étudiant la répartition des temps verbaux, en particulier la variation des temps entre les œuvres de jeunesse et celles plus récentes. De plus, nous examinerons l'utilisation des pronoms dans chaque œuvre, en nous concentrant sur les pronoms de première personne ``je'', les pronoms de troisième personne ``elle'', ``il'', ``elles'', ``ils'', ainsi que les pronoms représentant des groupes ou des interactions, tels que ``nous'' et ``vous'', en étudiant leur fréquence d'apparition. Nous analyserons également la distribution de ces pronoms dans les œuvres plus anciennes et plus récentes. Par ailleurs, nous évaluerons le nombre moyen de mots et de caractères par œuvre, afin de mieux comprendre la structure des phrases dans chaque texte. Enfin, nous introduirons l'indice TTR (Type-Token Ratio), qui analyse le rapport entre le nombre de mots différents et le nombre total de mots, pour observer la richesse lexicale de chaque œuvre et comparer cette richesse entre les œuvres de jeunesse et les plus récentes.\\

Deuxièmement, au niveau de la structure syntaxique, nous procéderons à un marquage morphosyntaxique des tokens pour observer la proportion de différents types de mots, notamment les adverbes, les adjectifs, les conjonctions de coordination, les conjonctions subordonnées et les noms, et analyser les différences structurelles entre les œuvres de jeunesse et les plus récentes.\\

Troisièmement, au niveau des figures de style, nous nous concentrerons sur la densité des répétitions et des parallélismes dans chaque œuvre, en comparant leur fréquence d'utilisation dans les œuvres anciennes et récentes. Étant donné que d'autres figures de style, comme les métaphores et les comparaisons, sont plus difficiles à quantifier, notre étude se concentrera principalement sur ces deux figures.\\

Grâce à ces analyses, nous obtiendrons des résultats quantitatifs sur les trois dimensions : lexicale, syntaxique et stylistique pour chaque œuvre, et nous pourrons observer les différences entre les œuvres de jeunesse et les œuvres plus récentes. Cela nous permettra de répondre aux questions suivantes :
\begin{enumerate}
    \item Les œuvres d'Annie Ernaux présentent-elles réellement une caractéristique de ``binarité'' absolue ?
    \item Les différences entre ses œuvres de jeunesse et ses œuvres récentes sont-elles réellement significatives ?
    \item Peut-on identifier des éléments syntaxiques et stylistiques qui traversent l'ensemble de ses œuvres ?
    \item Devons-nous affiner, nuancer, ou au contraire confirmer la transformation esthétique qu'Annie Ernaux évoque à partir de sa quatrième œuvre, afin de redéfinir plus concrètement ce qu'elle entend par ``écriture plate'' ?
\end{enumerate}


Enfin, en nous appuyant sur la définition de l' ``écriture plate'' que nous aurons développée à partir de cette étude, nous analyserons les valeurs du TTR et les fréquences moyennes des œuvres du corpus externe, pour observer si nous pouvons y reconnaître des caractéristiques de ce style d'écriture , et déterminer si il est spécifique à Annie Ernaux.

\section{Outils de visualisation numérique utilisés}

Dans le cadre de cette étude, nous avons utilisé divers outils de visualisation afin de représenter graphiquement les données textuelles analysées. Ces outils facilitent la compréhension des tendances linguistiques, des caractéristiques de distribution et de leur évolution au sein des œuvres étudiées :\\


\textbf{Diagrammes en barres (bar plots)} : Ce type de graphique permet de comparer visuellement les fréquences ou les moyennes entre différentes catégories. Par exemple, nous avons utilisé des diagrammes en barres pour représenter la fréquence d’utilisation des pronoms, la répartition des temps verbaux, ainsi que la moyenne du nombre de mots par œuvre.\\

\textbf{Boîtes à moustaches (box plots)} : Les diagrammes en boîte offrent une représentation synthétique de la distribution d’un ensemble de données en indiquant la médiane, les quartiles et les éventuels points aberrants. Ils sont particulièrement adaptés pour comparer la variabilité de certains indicateurs linguistiques (comme la valeur du TTR ou la richesse lexicale) entre les œuvres de jeunesse et les œuvres plus récentes.\\

\textbf{Nuages de mots (word clouds)} : Les nuages de mots sont une forme de visualisation fondée sur la fréquence des mots, où la taille d’un mot est proportionnelle à sa fréquence d’apparition dans le texte. Ils permettent une première exploration qualitative des lexiques dominants dans chaque œuvre ou groupe d’œuvres, et peuvent révéler des thèmes récurrents ou des évolutions lexicales.\\

En complément, nous avons également utilisé le \textbf{TTR (Type-Token Ratio, ratio types-tokens)}, un indicateur important permettant de mesurer la richesse lexicale d’un texte, largement utilisé en linguistique. Le TTR se calcule en divisant le nombre de mots différents (\textit{types}) par le nombre total de mots (\textit{tokens}) dans un texte. Les \textit{types} désignent tous les mots distincts (chaque mot n’est compté qu’une seule fois), tandis que les \textit{tokens} incluent l’ensemble des mots, y compris les répétitions.\\

La valeur du TTR reflète la diversité lexicale d’un texte : un TTR élevé indique une plus grande diversité de vocabulaire, tandis qu’un TTR faible traduit une utilisation plus répétitive du lexique. Dans notre étude, nous avons utilisé le TTR pour comparer la richesse lexicale des différentes œuvres, en nous intéressant particulièrement aux différences entre les œuvres de jeunesse et les œuvres récentes d’Annie Ernaux. Cette analyse permet de mettre en lumière l’évolution stylistique de l’autrice : par exemple, une diminution du TTR pourrait indiquer une simplification du langage et un style plus direct, associé à l’esthétique d’une « écriture plate » ; à l’inverse, une augmentation du TTR pourrait signaler un enrichissement lexical et une complexité croissante du style.
