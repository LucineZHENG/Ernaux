Depuis la publication de son premier roman \textit{Les Armoires vides} en 1974, la quête centrale d’Annie Ernaux est restée celle de l’expression authentique de toutes les dimensions de l’expérience vécue. Après la parution de trois romans de jeunesse (\textit{Les Armoires vides}, \textit{Ce qu’ils disent ou rien}, \textit{La Femme gelée}), Ernaux s’est tournée résolument vers une écriture mêlant autobiographie sociale, forme diaristique et observation extérieure. En 1984, elle obtient le prix Renaudot pour \textit{La Place}, et établit progressivement un style d’écriture fondé sur une langue épurée, directe et dénuée d’ornement, pour retranscrire la vie intime, la souffrance et la mémoire collective. Ce style, qui s’impose comme une des marques de la littérature française des années 1980, est aujourd’hui reconnu comme une forme singulière, qualifiée d’\textit{écriture plate}. Toutefois, sa définition précise dans le champ du traitement automatique du langage (TAL) reste encore largement inexplorée.\\

Cette étude vise donc à caractériser systématiquement, à l’aide de méthodes d’analyse quantitative, les traits linguistiques (lexicaux, syntaxiques et rhétoriques) propres à chaque œuvre d’Ernaux, afin de rendre compte de l’évolution stylistique de son écriture, depuis les expérimentations des débuts jusqu’à la maturité formelle de la période tardive. Par ailleurs, cette recherche pose la question suivante : l’\textit{écriture plate} est-elle une invention esthétique propre à Ernaux, ou bien reflète-t-elle une tendance stylistique partagée par les écrivains français de la seconde moitié du XX\textsuperscript{e} siècle ? \\

Nous organisons la suite de cette contribution de la façon suivante : dans la section 2, nous présentons brièvement une revue des travaux utilisant des outils numériques pour l’identification des caractéristiques stylistiques dans les œuvres littéraires ; la section 3 décrit notre corpus, notre méthodologie et les outils de visualisation numériques; les sections 4 et 5 exposent et analysent en détail les résultats de nos expérimentations ; enfin, la section 6 synthétise les conclusions de notre étude et propose des pistes pour de futures recherches.
