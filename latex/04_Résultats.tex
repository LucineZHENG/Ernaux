\section{A. Corpus interne}

\subsubsection{1) Analyse lexicale}

Nous avons commencé par analyser le corpus interne au niveau lexical. Les résultats obtenus sont présentés comme suit :

\begin{itemize}[label=·]
    \item Liste des mots-clés
    \item Nuage de mots
    \item Catégorisation thématique
    \item Répartition des temps verbaux
    \item Répartition des pronoms
    \item Répartition des mots moyens et des caractères moyens (présentés par des diagrammes en barres et des boîtes à moustaches)
    \item Répartition du TTR
\end{itemize}

\subsubsection{2) Analyse syntaxique}

Ensuite, l'analyse des structures syntaxiques a permis de repérer les distributions des types de mots suivants :

\begin{itemize}[label=·]
    \item Adjectifs (Adj)
    \item Adverbes (Adv)
    \item Conjonctions de coordination (Cconj)
    \item Conjonctions subordonnées (Sconj)
    \item Noms (Noun)
\end{itemize}

\subsubsection{3) Analyse stylistique}

Enfin, l'analyse stylistique a été réalisée en se concentrant sur les figures de style suivantes :

\begin{itemize}[label=·]
    \item Répartition des répétitions
    \item Répartition des parallélismes
\end{itemize}

\section{Définition des indicateurs quantitatifs de l'écriture plate}

D'après les résultats obtenus, nous avons déterminé les indicateurs quantitatifs pour l'écriture plate comme suit :

\begin{itemize}[label=·]
    \item Moyenne du TTR : 0.4 
    
    En effet, la moyenne du TTR pour les œuvres de jeunesse d'Ernaux est de 0.3545, tandis que celle des œuvres récentes est de 0.4738. Ainsi, nous avons choisi de fixer l'intervalle de TTR entre 0.35 et 0.45, soit une valeur centrale de 0.4.
    
    \item Moyenne du nombre de mots : 18
    
    Le nombre moyen de mots par œuvre dans les œuvres de jeunesse d'Ernaux est généralement supérieur à 18, tandis que celui des œuvres récentes est souvent inférieur à 18. Ce phénomène est particulièrement marqué dans des œuvres à style concis comme \textit{Passion simple}. Nous avons donc fixé le seuil du nombre moyen de mots à 18.
\end{itemize}

\section{B. Corpus externe}

Dans le corpus externe, nous avons également analysé la répartition du TTR et du nombre moyen de mots pour chaque œuvre :

\begin{itemize}[label=·]
    \item Répartition du TTR pour chaque œuvre
    \item Répartition du nombre moyen de mots pour chaque œuvre
    \item Comparaison de la distribution du TTR et du nombre moyen de mots entre les œuvres d'Ernaux et celles d'autres auteurs
\end{itemize}
\newpage

  










